\documentclass{article}

\usepackage{color,courier,fancyhdr,listings,parcolumns,verbatim}
\usepackage[top=1in, bottom=1in, left=1in, right=1in]{geometry}

\title{TCR}
\author{git diff solution (Jens Heuseveldt, Ludo Pulles, Peter Ypma)}

\lstset{
	basicstyle=\footnotesize\ttfamily,
	breaklines=true,
	commentstyle=\color[rgb]{0.5,0.5,0.5},
	frame=lines,
	keywordstyle=\color{blue},
	language=C++,
	numbers=left,
	numbersep=5pt,
	numberstyle=\tiny\color[rgb]{0.6,0.6,0.6},
	otherkeywords={ios_base, cin, cout, pii, LLINF, INF, NUM},
	showstringspaces=false,
	stringstyle=\color[rgb]{0.6,0.3,0},
	tabsize=4,
	xleftmargin=\parindent
}

\pagestyle{fancy}
\lhead{Utrecht University}
\rhead{git diff solution}

\begin{document}

% \maketitle
\begin{center}
	\makeatletter
	\textbf{\@title} \\
	\today \\
	\emph{\@author}
	\makeatother
\end{center}

% \tableofcontents
% \newpage

\begin{lstlisting}
#include<bits/stdc++.h>

#define x first
#define y second

using namespace std;

typedef long long ll;
typedef pair<int, int> pii;
typedef pair<ll, ll> pll;
typedef vector<int> vi;

const int INF = 2147483647; // (1 << 30) - 1 + (1 << 30)
const ll LLINF = 9223372036854775807LL; // (1LL << 62) - 1 + (1LL << 62)
const double pi = acos(-1.0);

// lambda-expression: [] (args) -> retType { body }

void run() {}

int main() {
	ios_base::sync_with_stdio(false);
	cin.tie(NULL);
	cout.tie(NULL);
	// cerr << boolalpha;
	(cout << fixed).precision(10);
	int ntc;
	cin >> ntc;
	while (ntc--) { run(); }
	return 0;
}

\end{lstlisting}

Two prime numbers: $\mathit{982451653}$, $\mathit{81253449}$

\subsection{Covering problems}

\begin{center}
	\emph{Minimum edge cover$\iff$Maximum independent set}
\end{center}

\textbf{Matching} \\
A set of edges without common vertices \textit{(Maximum is the \textbf{largest} such set, maximal is a set which you cannot add more edges to without breaking the property)}.

\textbf{Minimum Vertex Cover} \\
A set vertices (cover) such that each edge in the graph is incident to at least one vertex of the set.

\textbf{Minimum Edge Cover} \\
A set of edges (cover) such that every vertex is incident to at least one edge of the set.

\textbf{Maximum Independent Set} \\
A set of vertices in a graph such that no two of them are adjacent.

\textbf{K\"{o}nig's theorem} \\
In any bipartite graph, the number of edges in a maximum matching equals the number of vertices in a minimum vertex cover

\section{Mathematics}

\begin{lstlisting}

int abs(int x) { return x > 0 ? x : -x; }
int sign(int x) { return (x > 0) - (x < 0); }

// greatest common divisor
ll gcd(ll a, ll b) {
	while (b) {
		ll c = a % b;
		a = b; b = c;
	}
	return a;
}

// least common multiple
ll lcm(ll a, ll b) { return a / gcd(a, b) * b; }

// ax + by = gcd(a, b)
ll egcd(ll a, ll b, ll &x, ll &y) {
	ll r = b, rr = a, s = 0, ss = 1, t = 1, tt = 0, tmp;
	while (r) {
		ll q = rr / r;
		tmp = r; r = rr - q * r; rr = tmp;
		tmp = s; s = ss - q * s; ss = tmp;
		tmp = t; t = tt - q * t; tt = tmp;
	}
	x = ss; y = tt;
	return rr; // gcd
}

ll mod(ll a, ll m) {
	a %= m;
	return (a < 0) ? a + m : a;
}

const pll INVALID_CRT(0, -1);

// x = a0 mod b0 = a1 mod b1
pll crt(ll a1, ll m1, ll a2, ll m2) {
	ll x, y, gcd = egcd(m1, m2, x, y);
	if (a1 % gcd != a2 % gcd) return INVALID_CRT;
	return pll(mod(x * a2 * m1 + y * a1 * m2, m1 * m2) / gcd, m1 / gcd * m2);
}

// totient function for values 1 to N
int phi[N];

void sievePhi() {
	for (int i = N; i--; ) phi[i] = i;
	for (int i = 2; i <= N; i++)
		if (phi[i] == i)
			for (int j = i; j <= N; j += i)
				phi[j] -= phi[j] / i * (i - 1);
}

// modular exponentation: r = b^e mod m
ll modpow(ll b, ll e, ll m) {
	ll r = 1;
	while (e) {
		if (e & 1) r = (r * b) % m;
		e >>= 1;
		b = (b * b) % m;
	}
	return r;
}
\end{lstlisting}

\section{Datastructures}
\subsection{Segment tree $\mathcal{O}(\log n)$}

\begin{lstlisting}
typedef /* Tree element */ S;
const int n = 1 << 20;
S t[2 * n];

// sum segment tree
S combine(S l, S r) { return l + r; }
// max segment tree
S combine(S l, S r) { return max(l, r); }

void build() {
	for (int i = n; --i > 0; )
		t[i] = combine(t[2 * i], t[2 * i + 1]);
}

// set value v on position p
void update(int p, int v) {
	for (t[p += n] = v; p /= 2; )
		t[p] = combine(t[2 * p], t[2 * p + 1]);
}

// sum on interval [l, r)
S query(int l, int r) {
	S resL, resR;
	for (l += n, r += n; l < r; l /= 2, r /= 2) {
		if (l & 1) resL = combine(resL, t[l++]);
		if (r & 1) resR = combine(t[--r], resR);
	}
	return combine(resL, resR);
}

\end{lstlisting}

\subsection{Binary Indexed Tree $\mathcal{O}(\log n)$}

\begin{lstlisting}
int bit[MAXN];

// arr[idx] += val
void update(int idx, int val) {
	while (idx < MAXN) bit[idx] += val, idx += idx & -idx;
}

// returns sum of arr[i], where i: [1, idx]
int query(int idx) {
	int ret = 0;
	while (idx) ret += bit[idx], idx -= idx & -idx;
	return ret;
}

\end{lstlisting}

\subsection{Trie}

\begin{lstlisting}
struct trie {
	bool word;
	trie **child;

	trie() : word(false), child() {
		child = new trie*[26];
		for (int i = 26; i--; ) child[i] = NULL;
	}

	void addWord(const string &str)
	{
		trie *cur = this;
		for (char ch : str) {
			int idx = ch - 'a';
			if (cur->child[idx] == NULL)
				cur->child[idx] = new trie();
			cur = cur->child[idx];
		}
		cur->word = true;
	}

	bool isWord(const string &str)
	{
		trie *cur = this;
		for (char ch : str) {
			int idx = ch - 'a';
			if (cur->child[idx] == NULL) return false;
			cur = cur->child[idx];
		}
		return cur->word;
	}
};
\end{lstlisting}

\subsection{Disjoint-Set / Union-Find $\mathcal{O}(\alpha (n))$}

\begin{lstlisting}
int par[MAXN], rnk[MAXN];

void uf_init(int n) {
	fill_n(par, n, -1);
	fill_n(rnk, n, 0);
}

int uf_find(int v) {
	return par[v] < 0 ? v : par[v] = uf_find(par[v]);
}

void uf_union(int a, int b) {
	if ((a = uf_find(a)) == (b = uf_find(b))) return;
	if (rnk[a] < rnk[b]) swap(a, b);
	if (rnk[a] == rnk[b]) rnk[a]++;
	par[b] = a;
}
\end{lstlisting}

\section{Graph Algorithms}
\subsection{Maximum matching $\mathcal{O}(n m)$}

This problem could be solved with a flow algorithm like Dinic's algorithm which runs in $\mathcal{O} (\sqrt{V} E)$, too.

\begin{lstlisting}
bool vis[nodesRight]; // vis[rightnodes]
int par[nodesRight]; // par[rightnode] = leftnode
vector<int> adj[nodesLeft]; // adj[leftnode][i] = rightnode

bool match(int cur) {
	for (int nxt : adj[cur]) {
		if (vis[nxt]) continue;
		vis[nxt] = true;
		if (par[nxt] == -1 || match(par[nxt])) {
			par[nxt] = cur;
			return true;
		}
	}
	return false;
}

int matches = 0;
for (int i = 0; i < nodesLeft; i++) {
	memset(vis, false, sizeof(vis));
	if (match(i)) matches++;
}
\end{lstlisting}

\subsection{Strongly Connected Components $\mathcal{O}(V + E)$}

\begin{lstlisting}
vector<vi> adj; // adjacency matrix
vi index, lowlink; // lowest index reachable
stack<int> tarjanStack;
vector<bool> inStack; // true iff in tarjanStack
int newId;
vector<vi> scc; // Output: collection of vertex sets

void tarjan(int v) {
	index[v] = lowlink[v] = newId++;
	tarjanStack.push(v);
	inStack[v] = true;
	for (int w : adj[v]) {
		if (index[w] == 0) {
			tarjan(w);
			lowlink[v] = min(lowlink[v], lowlink[w]);
		} else if (inStack[w]) {
			lowlink[v] = min(lowlink[v], index[w]);
		}
	}
	
	if (lowlink[v] == index[v]) {
		scc.push_back(vi());
		int w;
		do {
			w = tarjanStack.top();
			scc.back().push_back(w);
			inStack[w] = false;
			tarjanStack.pop();
		} while (w != v);
	}
}

int findSCC() {
	newId = 1;
	index.clear(); index.resize(n + 1, 0);
	lowlink.clear(); lowlink.resize(n + 1, 0);
	inStack.clear(); inStack.resize(n + 1, false);
	while (!tarjanStack.empty()) tarjanStack.pop();
	scc.clear();

	for (int i = 0; i < n; i++) {
		if (index[i] == 0) tarjan(i);
	}
	return scc.size();
}

\end{lstlisting}

\subsection{Cycle Detection $\mathcal{O}(V + E)$}

\begin{lstlisting}
vector<vi> adj; // assumes bidirected graph, adjust accordingly
vector<bool> vis(MAXN, false);
vector<int> par(MAXN, -1);

bool cycle_detection() {
	stack<int> s;
	s.push(0);
	vis[0] = true;
	while(!s.empty()) {
		int cur = s.top();
		s.pop();
		for(int i : adj[cur]) {
			if(vis[i] && par[cur] != i) return true;
			s.push(i);
			par[i] = cur;
			vis[i] = true;
		}
	}
	return false;
} 

\end{lstlisting}

\subsection{Shortest path}
\subsubsection{BFS $\mathcal{O}(V + E)$}

\begin{lstlisting}
int n, dist[MAXN];
vector<int> edges[MAXN]; // (to, cost)

// faster than dijkstra when all edge costs are the same
int bfs(int from, int to) {
fill_n(dist, n, -1);
dist[from] = 0;

queue<int> q;
q.push(from);
while (!q.empty()) {
int cur = q.front();
q.pop();
for (int nxt : edges[cur]) {
if (dist[nxt] >= 0) {
dist[nxt] = dist[cur] + 1;
if (nxt == to) return dist[nxt];
q.push(nxt);
}
}
}
return -1;
}

\end{lstlisting}

\subsubsection{Dijkstra $\mathcal{O}(E + V \log V)$}

\begin{lstlisting}
int n; // number of nodes

vector<pii> edges[MAXN]; // (to, cost)
int dist[MAXN];
bool vis[MAXN];

void dijkstra() {
fill_n(vis, n, false);
priority_queue<pii, vector<pii>, greater<pii> > q; // (dist, id)
q.push(pii(0, 0));

while (!q.empty()) {
pii v = q.top();
q.pop();

if (vis[v.second]) continue;
vis[v.second] = true;

for (const pii e : edges[v.second]) {
q.push(pii(v.first + e.second, e.first));
}
dist[v.second] = v.first;
}
}
\end{lstlisting}

\subsubsection{Floyd-Warshall $\mathcal{O}(V^{3})$}

\begin{lstlisting}
int n = 100, d[MAXN][MAXN];
for (int i = 0; i < n; i++) fill_n(d[i], n, INF / 3);
// set direct distances from i to j in d[i][j] (and d[j][i])
for (int i = 0; i < n; i++)
for (int j = 0; j < n; j++)
for (int k = 0; k < n; k++)
d[j][k] = min(d[j][k], d[j][i] + d[i][k]);

\end{lstlisting}

\subsubsection{Bellman Ford $\mathcal{O}(V E)$}

\begin{lstlisting}
vector< pair<pii,int> > edges; // ((from, to), cost)
vector<int> dist(MAXN);

bool bellman_ford(int source) {
	for (int i = 0; i < MAXN; i++) dist[i] = INF / 3;
	dist[source] = 0;
	
	bool updated;
	int loops = 0;
	do {
		updated = false;
		for (auto e : edges) {
			int alt = dist[e.first.first] + e.second;
			if (alt < dist[e.first.second]) {
				dist[e.first.second] = alt;
				updated = true;
			}
			// if undirected graph:
			int alt = dist[e.first.second] + e.second;
			if (UNDIRECTED && alt < dist[e.first.first]) {
				dist[e.first.first] = alt;
				updated = true;
			}
		}
	} while(updated && loops < n);
	return loops < n; // loops >= n: negative cycles
}

\end{lstlisting}

\subsection{Max-flow min-cut}
\subsubsection{Dinic's Algorithm $\mathcal{O}(V^{2} E)$}

\begin{lstlisting}
// http://www.slideshare.net/KuoE0/acmicpc-dinics-algorithm
struct edge {
	int to, rev;
	ll cap, flow;
	edge(int t, int r, ll c) : to(t), rev(r), cap(c), flow(0) {}
};

int s, t, level[MAXN]; // s = source, t = sink
vector<edge> g[MAXN];

bool dinic_bfs() {
	fill_n(level, MAXN, 0);
	level[s] = 1;

	queue<int> q;
	q.push(s);
	while (!q.empty()) {
		int cur = q.front();
		q.pop();
		for (edge e : g[cur]) {
			if (level[e.to] == 0 && e.flow < e.cap) {
				level[e.to] = level[cur] + 1;
				q.push(e.to);
			}
		}
	}
	return level[t] != 0;
}

ll dinic_dfs(int cur, ll maxf) {
	if (cur == t) return maxf;

	ll f = 0;
	bool isSat = true;
	for (edge &e : g[cur]) {
		if (level[e.to] != level[cur] + 1 || e.flow >= e.cap)
			continue;
		ll df = dinic_dfs(e.to, min(maxf - f, e.cap - e.flow));
		f += df;
		e.flow += df;
		g[e.to][e.rev].flow -= df;
		isSat &= e.flow == e.cap;
		if (maxf == f) break;
	}
	if (isSat) level[cur] = 0;
	return f;
}

ll dinic_maxflow() {
	ll f = 0;
	while (dinic_bfs()) f += dinic_dfs(s, LLINF);
	return f;
}

void add_edge(int fr, int to, ll cap) {
	g[fr].push_back(edge(to, g[to].size(), cap));
	g[to].push_back(edge(fr, g[fr].size() - 1, 0));
}
\end{lstlisting}

\subsection{Min-cost max-flow}

\begin{lstlisting}
struct edge {
	// to, rev, flow, capacity, weight
	int t, r;
	ll f, c, w;
	edge(int _t, int _r, ll _c, ll _w) : t(_t), r(_r), f(0), c(_c), w(_w) {}
};

int n, par[MAXN];
vector<edge> adj[MAXN];
ll dist[MAXN];

bool findPath(int s, int t)
{
	fill_n(dist, n, LLINF);
	fill_n(par, n, -1);

	priority_queue< pii, vector<pii>, greater<pii> > q;
	q.push(pii(dist[s] = 0, s));

	while (!q.empty()) {
		int d = q.top().first, v = q.top().second;
		q.pop();
		if (d > dist[v]) continue;

		for (edge e : adj[v]) {
			if (e.f < e.c && d + e.w < dist[e.t]) {
				q.push(pii(dist[e.t] = d + e.w, e.t));
				par[e.t] = e.r;
			}
		}
	}
	return dist[t] < INF;
}

pair<ll, ll> minCostMaxFlow(int s, int t)
{
	ll cost = 0, flow = 0;
	while (findPath(s, t)) {
		ll f = INF, c = 0;
		int cur = t;
		while (cur != s) {
			const edge &rev = adj[cur][par[cur]], &e = adj[rev.t][rev.r];
			f = min(f, e.c - e.f);
			cur = rev.t;
		}
		cur = t;
		while (cur != s) {
			edge &rev = adj[cur][par[cur]], &e = adj[rev.t][rev.r];
			c += e.w;
			e.f += f;
			rev.f -= f;
			cur = rev.t;
		}
		cost += f * c;
		flow += f;
	}
	return pair<ll, ll>(cost, flow);
}

inline void addEdge(int from, int to, ll cap, ll weight)
{
	adj[from].push_back(edge(to, adj[to].size(), cap, weight));
	adj[to].push_back(edge(from, adj[from].size() - 1, 0, -weight));
}
\end{lstlisting}

\subsection{Minimal Spanning Tree}
\subsubsection{Prim $\mathcal{O}((E + V) \log V)$}

\begin{lstlisting}
// minimum spanning forest actually...
vector<pii> edges[MAXN]; // or set
int dist[MAXN];
bool done[MAXN];

ll prim(int n) {
	fill_n(dist, n, INF);
	fill_n(done, n, false);
	ll ret = 0, trees = 0;
	set<pii> q; // (to MST, vertex)
	for (int i = 0; i < n; i++) {
		if (done[i]) continue;
		trees++;
		q.insert(pii(dist[i] = 0, i));
		while (!q.empty()) {
			ret += q.begin()->first;
			int cur = q.begin()->second;
			q.erase(q.begin());
			done[cur] = true;
			for (pii pr : edges[cur]) {
				if (!done[pr.first] && pr.second < dist[pr.first]) {
					q.erase(pii(dist[pr.first], pr.first));
					dist[pr.first] = pr.second;
					q.insert(pii(dist[pr.first], pr.first));
				}
			}
		}
	}
	// if (trees > 1) return -1; // forest
	return ret;
}
\end{lstlisting}

\subsubsection{Kruskal $\mathcal{O}(E \log V)$}

\begin{lstlisting}
struct edge {
	int x, y, s;
	void read() { cin >> x >> y >> s; }
};

edge edges[MAXM];

int kruskal(int n, int m) {
	uf_init(n);
	sort(edges, edges + m, [] (const edge &a, const edge &b)
		-> bool { return a.s > b.s; });
	ll ret = 0;
	while (m--) {
		if (uf_find(edges[m].x) != uf_find(edges[m].y)) {
			ret += edges[m].s;
			uf_union(edges[m].x, edges[m].y);
		}
	}
	return ret;
}

\end{lstlisting}

\section{String algorithms}
\subsection{Z-algorithm $\mathcal{O}(n)$}

\begin{lstlisting}
// z[i] = length of longest substring starting from s[i],
// which is also a prefix of s.
vector<int> z_function(const string &s) {
	int n = (int) s.length();
	vector<int> z(n);
	for (int i = 1, l = 0, r = 0; i < n; ++i) {
		if (i <= r)
			z[i] = min (r - i + 1, z[i - l]);
		while (i + z[i] < n && s[z[i]] == s[i + z[i]])
			++z[i];
		if (i + z[i] - 1 > r)
			l = i, r = i + z[i] - 1;
	}
	return z;
}
\end{lstlisting}

\subsection{Longest Common Subsequence $\mathcal{O}(n^{2})$}
\textsc{Substring}: \textit{consecutive characters}\,!!!

\begin{lstlisting}
int table[STR_SIZE][STR_SIZE]; // DP problem

int lcs(const string &w1, const string &w2) {
	int n1 = w1.size(), n2 = w2.size();
	for (int i = 0; i <= n1; i++) table[i][0] = 0;
	for (int j = 0; j <= n2; j++) table[0][j] = 0;
	
	for (int i = 1; i < n1; i++) {
		for (int j = 1; j < n2; j++) {
			table[i][j] = w1[i - 1] == w2[j - 1] ? 
				(table[i - 1][j - 1] + 1) :
				max(table[i - 1][j], table[i][j - 1]);
		}
	}
	return table[n1][n2];
}

// backtrace
string getLCS(const string &w1, const string &w2) {
	int i = w1.size(), j = w2.size();
	string ret = "";
	while (i > 0 && j > 0) {
		if (w1[i - 1] == w2[j - 1]) ret += w1[--i], j--;
		else if (table[i][j - 1] > table[i - 1][j]) j--;
		else i--;
	}
	reverse(ret.begin(),ret.end());
	return ret;
}
\end{lstlisting}

\subsection{Levenshtein Distance $\mathcal{O}(n^{2})$}

\begin{lstlisting}
int costs[MAX_SIZE][MAX_SIZE]; // DP problem

int levDist(const string &w1, const string &w2) {
	int n1 = w1.size(), n2 = w2.size();
	for (int i = 0; i <= n1; i++) costs[i][0] = i; // removal
	for (int j = 0; j <= n2; j++) costs[0][j] = j; // insertion
	for (int i = 1;i <= n1; i++) {
		for (int j = 1; j <= n2; j++) {
			costs[i][j] = min(
				min(costs[i - 1][j] + 1, costs[i][j - 1] + 1),
				costs[i - 1][j - 1] + (w1[i - 1] != w2[j - 1])
			);
		}
	}
	return costs[n1][n2];
}
\end{lstlisting}

\subsection{Knuth-Morris-Pratt algorithm $\mathcal{O}(N + M)$}

\begin{lstlisting}
int kmp_search(const string &word, const string &text) {
	int n = word.size();
	vector<int> table(n + 1, 0);
	for (int i = 1, j = 0; i < n; ) {
		if (word[i] == word[j]) {
			table[++i] = ++j;
		} else if (j > 0) {
			j = table[j];
		} else i++;
	}
	int matches = 0;
	for (int i = 0, j = 0; i < text.size(); ) {
		if (text[i] == word[j]) {
			i++;
			if (++j == n) {
				matches++;
				// match at interval [i - j, i)
				j = table[j];
			}
		} else if (j > 0) j = table[j];
		else i++;
	}
	return matches;
}
\end{lstlisting}

\subsection{Aho-Corasick Algorithm $\mathcal{O}(N + \sum_{i=1}^{m} |S_i|)$}

\begin{lstlisting}

const int MAXP = 100, MAXLEN = 200, SIGMA = 26, MAXTRIE = MAXP * MAXLEN;

int npatterns;
string patterns[MAXP], S;

int wordIdx[MAXTRIE] = {}, to[MAXTRIE][SIGMA] = {}, sLink[MAXTRIE] = {}, dLink[MAXTRIE] = {}, nnodes = 1;

void ahoCorasick()
{
    // 1. Make a tree, 2. create sLinks and dLinks, 3. Walk through S
    fill_n(wordIdx, MAXTRIE, -1);
    for(int i = 0; i < npatterns; i++) {
        int cur = 0;
        for (char c : patterns[i]) {
            int idx = c - 'a';
            if(to[cur][idx] == 0) to[cur][idx] = nnodes++;
            cur = to[cur][idx];
        }
        wordIdx[cur] = i;
    }

    queue<int> q;
    q.push(0);
    while(!q.empty()) {
        int cur = q.front(); q.pop();
        for(int c = 0; c < SIGMA; c++) {
            if(to[cur][c]) {
                int sl = sLink[to[cur][c]] = cur == 0 ? 0 : to[sLink[cur]][c];
                // if all strings have equal length, remove this:
                dLink[to[cur][c]] = wordIdx[sl] >= 0 ? sl : dLink[sl];
                q.push(to[cur][c]);
            } else to[cur][c] = to[sLink[cur]][c];
        }
    }

    for (int cur = 0, i = 0, n = S.size(); i < n; i++) {
        int idx = S[i] - 'a';
        cur = to[cur][idx];
        // we have a match! (if g[i][j] >= 0)
        for (int hit = wordIdx[cur] >= 0 ? cur : dLink[cur]; hit; hit = dLink[hit]) {
            cerr << "Match for " << patterns[wordIdx[hit]] << " at " << (i + 1 - patterns[wordIdx[hit]].size()) << endl;
        }
    }
}

\end{lstlisting}

\section{Geometry}

\begin{lstlisting}
typedef double NUM; // either double or long long

struct pt {
	NUM x, y;
	
	pt() : x(0), y(0) {}
	pt(NUM _x, NUM _y) : x(_x), y(_y) {}
	pt(const pt &p) : x(p.x), y(p.y) {}
	
	pt operator*(NUM scalar) const {
		return pt(scalar * x, scalar * y); // scalar
	}
	NUM operator*(const pt &rhs) const {
		return x * rhs.x + y * rhs.y; // dot product
	}
	NUM operator^(const pt &rhs) const {
		return x * rhs.y - y * rhs.x; // cross product
	}
	pt operator+(const pt &rhs) const {
		return pt(x + rhs.x, y + rhs.y); // addition
	}
	pt operator-(const pt &rhs) const {
		return pt(x - rhs.x, y - rhs.y); // subtraction
	}
	bool operator==(const pt &rhs) const {
		return x == rhs.x && y == rhs.y;
	}
	bool operator!=(const pt &rhs) const {
		return x != rhs.x || y != rhs.y;
	}
};

// distance SQUARED from pt a to pt b
NUM sqDist(const pt &a, const pt &b) {
	return (a - b) * (a - b);
}

// distance SQUARED from pt a to line bc
double sqDistPointLine(pt a, pt b, pt c) {
	a = a - b;
	c = c - b;
	return (a ^ c) * (a ^ c) / (double)(c * c);
}

// distance SQUARED from pt a to line segment c
double sqDistPointSegment(pt a, pt b, pt c) {
	a = a - b;
	c = c - b;
	NUM dot = a * c, len = c * c;
	if (dot <= 0) return a * a;
	if (dot >= len) return (a - c) * (a - c);
	return a * a - dot * dot / ((double) len);
	// pt proj = c * dot / ((double) len);
}

bool between(NUM a, NUM b, NUM n) {
	return min(a, b) <= n && n <= max(a, b);
}
bool collinear(pt a, pt b, pt c) {
	return ((a - b) ^ (a - c)) == 0;
}

// point a on segment bc
bool pointOnSegment(pt a, pt b, pt c)
{
	return collinear(a, b, c) &&
		between(b.x, c.x, a.x) && between(b.y, c.y, a.y);
}

// Line segment a1 -- a2 intersects with b1 -- b2?
// returns 0: no, 1: yes at i1, 2: yes at i1 -- i2
int segmentsIntersect(pt a1, pt a2, pt b1, pt b2, pt &i1, pt &i2) {
	if (((a2 - a1) ^ (b2 - b1)) < 0) swap(a1, a2);
	// assert(a1 != a2 && b1 != b2);
	pt q = a2 - a1, r = b2 - b1, s = b1 - a1;
	NUM cross = q ^ r, c1 = s ^ r, c2 = s ^ q;
	if (cross == 0) {
		// line segments are parallel
		if ((q ^ s) != 0) return 0; // no intersection
		NUM v1 = s * q, v2 = (b2 - a1) * q, v3 = q * q;
		if (v2 < v1) swap(v1, v2), swap(b1, b2);

		if (v1 > v3 || v2 < 0) return 0; // intersection empty
		i1 = v2 > v3 ? a2 : b2;
		i2 = v1 < 0 ? a1 : b1;
		return i1 == i2 ? 1 : 2; // one point or overlapping
	} else { // cross > 0
		i1 = pt(a1) + pt(q) * (1.0 * c1 / cross); // needs double
		return 0 <= c1 && c1 <= cross && 0 <= c2 && c2 <= cross;
		// intersection inside segments
	}
}

// complete intersection check
int segmentsIntersect2(pt a1, pt a2, pt b1, pt b2, pt &i1, pt &i2) {
	if (a1 == a2 && b1 == b2) {
		i1 = a1;
		return a1 == b1;
	} else if (a1 == a2) {
		i1 = a1;
		return pointOnSegment(a1, b1, b2);
	} else if (b1 == b2) {
		i1 = b1;
		return pointOnSegment(b1, a1, a2);
	} else return segmentsIntersect(a1, a2, b1, b2, i1, i2);
}

// Returns TWICE the area of a polygon to keep it an integer
NUM polygonTwiceArea(const vector<pt> &polygon) {
	NUM area = 0;
	for (int i = 0, N = polygon.size(), j = N - 1; i < N; j = i++)
		area += polygon[i] ^ polygon[j];
	return abs(area);
}

// returns 0 outside, 1 inside, 2 on boundary
int pointInPolygon(pt p, const vector<pt> &polygon) {
	// Check corssings with horizontal semi-line through p to +x
	int crosscount = 0, N = polygon.size();
	for (int i = 0, j = N - 1; i < N; j = i++) {
		if (pointOnSegment(p, polygon[i], polygon[j])) return 2;

		// check if it crosses the vertical y = p.y line
		NUM l = (p.x - polygon[i].x)*(polygon[j].y - polygon[i].y);
		NUM r = (p.y - polygon[i].y)*(polygon[j].x - polygon[i].x);
		if (polygon[j].y > p.y) {
			if (polygon[i].y <= p.y && l < r) crosscount++;
		} else {
			if (polygon[i].y <= p.y && l > r) crosscount++;
		}
	}
	return crosscount & 1;
}

// Assumption: polygon has unique points
int pointInConvex(pt p, const vector<pt> &polygon) {
	// the cross product should always have the same sign,
	// when the point is inside the convex
	int N = polygon.size(), sgn = 0;
	bool onBoundary = false;
	for (int i = 0, j = N - 1; i < N; j = i++) {
		NUM cross = (polygon[j] - p) ^ (polygon[i] - p);
		if (cross == 0) onBoundary = true;
		else if (sgn == 0) sgn = sign(cross);
		else if (sgn != sign(cross)) return 0;
	}
	return onBoundary ? 2 : 1;
}

\end{lstlisting}

\subsection{Convex Hull $\mathcal{O}(n \log n)$}

\begin{lstlisting}
// output contains indices of the points on the hull
void convex_hull(const vector<pt> &pts, vector<int> &output) {
	output.clear();
	if (pts.size() < 3) {
		if (pts.size() >= 1) output.push_back(0);
		if (pts.size() >= 2) output.push_back(1);
		return;
	}

	unsigned int bestIndex = 0;
	NUM minX = pts[0].x, minY = pts[0].y;
	for(unsigned int i = 1; i < pts.size(); ++i) {
		if (pts[i].x < minX || (pts[i].x == minX && pts[i].y < minY)) {
			bestIndex = i;
			minX = pts[i].x;
			minY = pts[i].y;
		}
	}
	vector<int> ordered; //index into pts
	for(unsigned int i = 0; i < pts.size(); ++i) {
		if (i != bestIndex) ordered.push_back(i);
	}

	pt refr = pts[bestIndex];
	sort(ordered.begin(), ordered.end(), [&pts,&refr] (int a, int b) -> bool {
		NUM cross = (pts[a] - refr) ^ (pts[b] - refr);
		return cross != 0 ? cross > 0 : sqDist(refr, pts[a]) < sqDist(refr, pts[b]);
	});
	
	output.push_back(bestIndex);
	output.push_back(ordered[0]);
	output.push_back(ordered[1]);
	for(unsigned int i = 2; i < ordered.size(); ++i) {
		//NOTE: > INCLUDES and >= EXCLUDES points on the hull-line
		while (output.size() > 1 && ((pts[output[output.size() - 2]] - pts[output.back()]) ^ (pts[ordered[i]] - pts[output.back()])) > 0) {
			output.pop_back();
		}
		output.push_back(ordered[i]);
	}
	return;
}

\end{lstlisting}

\section{Miscellaneous}
\subsection{Binary search $\mathcal{O}(\log n)$}

\noindent\begin{minipage}{.45\textwidth}
	\begin{lstlisting}[title={Inclusive, Exclusive}]
bool test(int n);
	
int min = 0, max = n;
// assert(test(min) && !test(max));
while (max - min > 1) {
	int c = (min + max) / 2;
	if (test(c)) min = c;
	else         max = c;
}
return min;
	\end{lstlisting}
\end{minipage}\hfill
\begin{minipage}{.45\textwidth}
	\begin{lstlisting}[title={Inclusive, Inclusive}]
bool test(int n);
	
int lo = 0, hi = n - 1;
// assert(test(lo) && !test(hi + 1));
while (lo < hi) {
	int mid = (lo + hi + 1) / 2;
	if (test(mid)) lo = mid;
	else hi = mid - 1;
}
return lo;
	\end{lstlisting}
\end{minipage}

\subsection{Fast Fourier Transform $\mathcal{O}(n \log n)$}

Given two polynomials $A(x) = a_0 + a_1 x + \dots + a_{n/2} x^{n/2}$ and $B(x) = b_0 + b_1 x + \dots + b_{n/2} x^{n/2}$, FFT calculates all coefficients of $C(x) = A(x) \cdot B(x) = c_0 + c_1 x + \dots c_{n} x^{n}$.

\begin{lstlisting}

typedef complex<double> cpx;
const int logmaxn = 20, maxn = 1 << logmaxn;

cpx a[maxn] = {}, b[maxn] = {}, c[maxn];

void fft(cpx *src, cpx *dest)
{
	for (int i = 0, rep = 0; i < maxn; i++, rep = 0) {
		for (int j = i, k = logmaxn; k--; j >>= 1) rep = (rep << 1) | (j & 1);
		dest[rep] = src[i];
	}
	for (int s = 1, m = 1; m <= maxn; s++, m *= 2) {
		cpx r = exp(cpx(0, 2.0 * pi / m));
		for (int k = 0; k < maxn; k += m) {
			cpx cr(1.0, 0.0);
			for (int j = 0; j < m / 2; j++) {
				NUM t = cr * dest[k + j + m / 2];
				dest[k + j + m / 2] = dest[k + j] - t;
				dest[k + j] += t;
				cr *= r;
			}
		}
	}
}

void multiply()
{
	fft(a, c);
	fft(b, a);
	for (int i = 0; i < maxn; i++) b[i] = conj(a[i] * c[i]);
	fft(b, c);
	for (int i = 0; i < maxn; i++) c[i] = conj(c[i]) / (1.0 * maxn);
}

\end{lstlisting}

\subsection{Minimum Assignment (Hungarian Algorithm) $\mathcal{O}(n^{3})$}

\begin{lstlisting}
int n, m; // n rows, m columns
int a[MAXN + 1][MAXM + 1]; // matrix, 1-based
int minimum_assignment() {
	vector<int> u(n + 1), v(m + 1), p(m + 1), way(m + 1);

	for (int i = 1; i <= n; i++) {
		p[0] = i;
		int j0 = 0;
		vector<int> minv(m + 1, INF);
		vector<char> used(m + 1, false);
		do {
			used[j0] = true;
			int i0 = p[j0], delta = INF, j1;
			for (int j = 1; j <= m; j++)
				if (!used[j]) {
					int cur = a[i0][j] - u[i0] - v[j];
					if (cur < minv[j]) minv[j] = cur, way[j] = j0;
					if (minv[j] < delta) delta = minv[j], j1 = j;
				}
			for (int j = 0; j <= m; ++ j) {
				if(used[j]) u[p[j]] += delta, v[j] -= delta;
				else minv[j] -= delta;
			}
			j0 = j1;
		} while (p[j0] != 0);
		do {
			int j1 = way[j0];
			p[j0] = p[j1];
			j0 = j1;
		} while (j0);
	}

//	for (int j = 1; j <= m; ++ j) ans[p[j]] = j;
	return -v[0];
}
\end{lstlisting}

\end{document}

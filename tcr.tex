
\documentclass[8pt,a4paper,landscape,oneside]{amsart}
\usepackage{amsmath, amsthm, amssymb, amsfonts}
\usepackage[T1]{fontenc}
\usepackage[utf8]{inputenc}
\usepackage{booktabs}
\usepackage{caption}
\usepackage{color}
\usepackage{fancyhdr}
\usepackage{float}
\usepackage{fullpage}
\usepackage{xcolor}
\usepackage{textcomp}
%\usepackage{geometry}
% \usepackage[top=0pt, bottom=1cm, left=0.3cm, right=0.3cm]{geometry}
\usepackage[top=3pt, bottom=1cm, left=0.3cm, right=0.3cm]{geometry}
\usepackage{graphicx}
% \usepackage{listings}
\usepackage{subcaption}
\usepackage[scaled]{beramono}
\usepackage{titling}
\usepackage{datetime}
\usepackage{enumitem}
\usepackage{color,courier,fancyhdr,listings,parcolumns,verbatim,upquote}
\usepackage{multicol}

\newcommand{\subtitle}[1]{%
  \posttitle{%
    \par\end{center}
    \begin{center}\large#1\end{center}
    \vskip0.1em\vspace{-1em}}%
}

% Header/Footer
% \geometry{includeheadfoot}
%\fancyhf{}
\pagestyle{fancy}
\lhead{Reykjavík University}
\rhead{\thepage}
\cfoot{}
\setlength{\headheight}{15.2pt}
\setlength{\droptitle}{-20pt}
\posttitle{\par\end{center}}
\renewcommand{\headrulewidth}{0.4pt}
\renewcommand{\footrulewidth}{0.4pt}

% Math and bit operators
\DeclareMathOperator{\lcm}{lcm}
\newcommand*\BitAnd{\mathrel{\&}}
\newcommand*\BitOr{\mathrel{|}}
\newcommand*\ShiftLeft{\ll}
\newcommand*\ShiftRight{\gg}
\newcommand*\BitNeg{\ensuremath{\mathord{\sim}}}
\DeclareRobustCommand{\stirling}{\genfrac\{\}{0pt}{}}

\newenvironment{myitemize}
{ \begin{itemize}[leftmargin=.5cm]
    \setlength{\itemsep}{0pt}
    \setlength{\parskip}{0pt}
    \setlength{\parsep}{0pt}     }
{ \end{itemize}                  }

% Output Verbosity
\newif\ifverbose
\verbosetrue
% \verbosefalse

\title{TCR}
\author{git diff solution (Jens Heuseveldt, Ludo Pulles, Pim Spelier)}

\lstset{
	literate={~} {$\sim$}{1},
	basicstyle=\footnotesize\ttfamily,
	breaklines=true,
	commentstyle=\color[rgb]{0.5,0.5,0.5},
	frame=lines,
	keywordstyle=\color{blue},
	language=C++,
	numbers=left,
	numbersep=5pt,
	numberstyle=\tiny\color[rgb]{0.6,0.6,0.6},
	otherkeywords={cin, cout, cerr, pii, pll, PI, INF, LLINF, NUM},
	showstringspaces=false,
	stringstyle=\color[rgb]{0.6,0.3,0},
	tabsize=4,
	xleftmargin=\parindent
}

\pagestyle{fancy}
\lhead{Utrecht University}
\rhead{git diff solution}

\begin{document}

\begin{multicols*}{3}
\maketitle
\begin{center}
	\makeatletter
	\textbf{\@title} \\
%	\today \\
	\emph{\@author}
	\makeatother
\end{center}

\tableofcontents
% \newpage

\begin{center}
At the start of a contest, type this in a terminal:
\end{center}

\lstinputlisting{code/vim.sh}

\begin{center}
template.cpp
\end{center}

\lstinputlisting{code/T.cpp}

Prime numbers: $\mathit{982451653}$, $\mathit{81253449}$, $10^3 + \{-9,-3,9,13\}$, $10^6 + \{-17, 3, 33\}$, $10^9 + \{7,9,21,33,87\}$

\subsection{De winnende aanpak}

\begin{itemize}
	\setlength\itemsep{-.25em}
	\item Goed slapen \& een vroeg ritme hebben
	\item Genoeg drinken \& eten voor en tijdens de wedstrijd
	\item Een lijst van alle problemen met info waar het over gaat, en wie het goed kan oplossen
	\item Ludo moet {\huge\textbf{ALLE}} opgaves \textbf{goed} lezen
	\item Test de kleine voorbeeldgevallen
	\item Houd na 2 uur een pauze en overleg waar iedereen mee bezig is
	\item Maak zelf wat test-cases
	\item Typ de dingen uit de TCR, die je zeker nodig hebt, alvast in
	\item Als iemand niks te doen heeft, kan hij nodige dingen uit de TCR typen.
	\item We moeten ook een voorbeeld test-case voor TCR algoritmes hebben om te testen of het goed overgetypt is
	\item Bij geometrie moeten we om kunnen gaan met meerdere input manieren (voor bv. lijnen)
	\item Gebruik veel long long\textquotesingle s
\end{itemize}

\subsection{Wrong Answer}

\begin{enumerate}
	\setlength\itemsep{-.25em}
	\item Print de oplossing om te debuggen! Kijk ook naar andere (mogelijk makkelijkere) problemen.
	\item Bedenk zelf test-cases met \textbf{randgevallen}!
	\item Controleer op \textbf{overflow} (gebruik \textbf{OVERAL} long long, long double).
		\\ \textit{Kijk naar overflows in tussenantwoorden bij modulo.}
	\item Controleer de \textbf{precisie}.
	\item Controleer op \textbf{typo\textquotesingle s}.
	\item Loop de voorbeeldinput accuraat langs.
	\item Controller op off-by-one-errors (in indices of lus-grenzen)?
\end{enumerate}

\subsection{Detecting overflow}
These are GNU builtins, detect both over- and underflow. Returns a boolean upon failure, otherwise the result is present in \texttt{ref}. Follow the template:

\begin{lstlisting}
bool isOverflown = __builtin_[add|mul|sub]_overflow(a, b, \&res);
\end{lstlisting}

\subsection{Covering problems}

\begin{center}
	\emph{Minimum edge cover$\iff$Maximum independent set}
\end{center}

\begin{description}
	\setlength\itemsep{-.25em}
	\item[Matching]
		A set of edges without common vertices \textit{(Maximum is the \textbf{largest} such set, maximal is a set which you cannot add more edges to without breaking the property)}.
	\item[Minimum Vertex Cover]
		A set vertices (cover) such that each edge in the graph is incident to at least one vertex of the set.
	\item[Minimum Edge Cover]
		A set of edges (cover) such that every vertex is incident to at least one edge of the set.
	\item[Maximum Independent Set]
		A set of vertices in a graph such that no two of them are adjacent.
	\item[K\"{o}nig's theorem]
		In any bipartite graph, the number of edges in a maximum matching equals the number of vertices in a minimum vertex cover
\end{description}

% \begin{comment}
\subsection{Game theory}

A game can be reduced to Nim if it is a finite impartial game. Nim and its variants include:
\begin{description}
	\item[Nim] Let $X = \bigoplus_{i=1}^n x_i$, then $(x_i)_{i=1}^n$ is a winning position iff $X\neq 0$. Find a move by picking $k$ such that $x_k > x_k \oplus X$.
	\item[Mis\`{e}re Nim] Regular Nim, except that the last player to move \textit{loses}. Play regular Nim until there is only one pile of size larger than $1$, reduce it to $0$ or $1$ such that there is an odd number of piles.
	\item[Staricase Nim] Stones are moved down a staircase and only removed from the last pile. $(x_i)_{i=1}^n$ is an $L$-position if $(x_{2i-1})_{i=1}^{n/2}$ is (i.e. only look at odd-numbered piles).
	\item[Moore's Nim$_k$] The player may remove from at most $k$ piles (Nim $=$ Nim$_1$). Expand the piles in base $2$, do a carry-less addition in base $k+1$ (i.e. the number of ones in each column should be divisible by $k+1$).
	\item[Dim$^+$] The number of removed stones must be a divisor of the pile size. The Sprague-Grundy function is $k+1$ where $2^k$ is the largest power of $2$ dividing the pile size.
	\item[Aliquot game] Same as above, except the divisor should be proper (hence $1$ is also a terminal state, but watch out for size $0$ piles). Now the Sprague-Grundy function is just $k$.
	\item[Nim (at most half)] Write $n+1 = 2^my$ with $m$ maximal, then the Sprague-Grundy function of $n$ is $(y - 1) / 2$.
	\item[Lasker's Nim] Players may alternatively split a pile into two new non-empty piles. $g(4k+1) = 4k+1$, $g(4k+2) = 4k+2$, $g(4k+3) = 4k+4$, $g(4k+4) = 4k+3$ ($k\geq 0$).
	\item[Hackenbush on trees] A tree with stalks $(x_i)_{i=1}^n$ may be replaced with a single stalk with length $\bigoplus_{i=1}^n x_i$.
\end{description}
% \end{comment}

A useful identity: $\bigoplus_{x=0}^{a - 1} x = \{0, a - 1, 1, a\}[a \, \mathrm{mod} \, 4]$.

\section{Mathematics}

\lstinputlisting{code/math.cpp}

\section{Datastructures}
\subsection{Standard segment tree $\mathcal{O}(\log n)$}

\lstinputlisting{code/segment_tree.cpp}

\subsection{Binary Indexed Tree $\mathcal{O}(\log n)$}

Use one-based indices ($i > 0$)!

\lstinputlisting{code/bit.cpp}

\subsection{Disjoint-Set / Union-Find $\mathcal{O}(\alpha (n))$}

\lstinputlisting{code/dsu.cpp}

\section{Graph Algorithms}
% TODO: articulation points, and bridges!
\subsection{Maximum matching $\mathcal{O}(n m)$}

This problem could be solved with a flow algorithm like Dinic's algorithm which runs in $\mathcal{O} (\sqrt{V} E)$, too.

\lstinputlisting{code/maxmatch.cpp}

\subsection{Strongly Connected Components $\mathcal{O}(V + E)$}

\lstinputlisting{code/scc.cpp}

\subsubsection{2-SAT $\mathcal{O}(V+E)$}

Include \texttt{findSCC}.

\lstinputlisting{code/satsolver.cpp}

% \begin{comment}
\subsection{Cycle Detection $\mathcal{O}(V + E)$}

\lstinputlisting{code/findcycle.cpp}
% \end{comment}

\subsection{Shortest path}

% Hier zitten sowieso al heel veel fouten in, en het is niet flexibel voor begin-, en eindpunt
\subsubsection{Dijkstra $\mathcal{O}(E + V \log V)$}

\subsubsection{Floyd-Warshall $\mathcal{O}(V^{3})$}

\lstinputlisting{code/floydwarshall.cpp}

\subsubsection{Bellman Ford $\mathcal{O}(V E)$}

This is only useful if there are edges with weight $w_{i j} < 0$ in the graph.

\lstinputlisting{code/bellmanford.cpp}

\subsection{Max-flow min-cut}
\subsubsection{Dinic's Algorithm $\mathcal{O}(V^{2} E)$}

\begin{comment}
// http://www.slideshare.net/KuoE0/acmicpc-dinics-algorithm
\end{comment}

\lstinputlisting{code/dinic.cpp}

\subsection{Min-cost max-flow}

Find the cheapest possible way of sending a certain amount of flow through a flow network.

\lstinputlisting{code/mincostmaxflow.cpp}

\subsection{Minimal Spanning Tree}

% Kruskal is shorter...
\begin{comment}
\subsubsection{Prim $\mathcal{O}((E + V) \log V)$}

\lstinputlisting{code/mst_prim.cpp}
\end{comment}

\subsubsection{Kruskal $\mathcal{O}(E \log V)$}

\begin{comment}
\lstinputlisting{code/mst_kruskal.cpp}
\end{comment}

\section{String algorithms}

\subsection{Trie}

\lstinputlisting{code/trie.cpp}

\subsection{Z-algorithm $\mathcal{O}(n)$}

\lstinputlisting{code/z_function.cpp}

\subsection{Suffix array $\mathcal{O}(n \log^2{n})$}

This creates an array $P[0], P[1], \dots, P[n-1]$ such that the suffix $S[i \dots n]$ is the $P[i]^\textit{th}$ suffix of $S$ when lexicographically sorted.

\begin{lstlisting}
typedef pair<pii, int> tii;

const int maxlogn = 17, int maxn = 1 << maxlogn;

tii make_triple(int a, int b, int c) { return tii(pii(a, b), c); }

int p[maxlogn + 1][maxn];
tii L[maxn];

int suffixArray(string S) {
	int N = S.size(), stp = 1, cnt = 1;
	for (int i = 0; i < N; i++) p[0][i] = S[i];
	for (; cnt < N; stp++, cnt <<= 1) {
		for (int i = 0; i < N; i++) {
			L[i] = tii(pii(p[stp-1][i], i + cnt < N ? p[stp-1][i + cnt] : -1), i);
		}
		sort(L, L + N);
		for (int i = 0; i < N; i++) {
			p[stp][L[i].y] = i > 0 && L[i].x == L[i-1].x ? p[stp][L[i-1].y] : i;
		}
	}
	return stp - 1; // result is in p[stp - 1][0 .. (N - 1)]
}
\end{lstlisting}

\subsection{Longest Common Subsequence $\mathcal{O}(n^{2})$}
\textsc{Substring}: \textit{consecutive characters}\,!!!

\begin{lstlisting}
int dp[STR_SIZE][STR_SIZE]; // DP problem

int lcs(const string &w1, const string &w2) {
	int n1 = w1.size(), n2 = w2.size();
	for (int i = 0; i < n1; i++) {
		for (int j = 0; j < n2; j++) {
			if (i == 0 || j == 0) dp[i][j] = 0;
			else if (w1[i - 1] == w2[j - 1]) dp[i][j] = dp[i - 1][j - 1] + 1;
			else dp[i][j] = max(dp[i - 1][j], dp[i][j - 1]);
		}
	}
	return dp[n1][n2];
}

// backtrace
string getLCS(const string &w1, const string &w2) {
	int i = w1.size(), j = w2.size();
	string ret = "";
	while (i > 0 && j > 0) {
		if (w1[i - 1] == w2[j - 1]) ret += w1[--i], j--;
		else if (dp[i][j - 1] > dp[i - 1][j]) j--;
		else i--;
	}
	reverse(ret.begin(), ret.end());
	return ret;
}
\end{lstlisting}

\subsection{Levenshtein Distance $\mathcal{O}(n^{2})$}

Also known as the `Edit distance'.

\begin{lstlisting}
int dp[MAX_SIZE][MAX_SIZE]; // DP problem

int levDist(const string &w1, const string &w2) {
	int n1 = w1.size(), n2 = w2.size();
	for (int i = 0; i <= n1; i++) dp[i][0] = i; // removal
	for (int j = 0; j <= n2; j++) dp[0][j] = j; // insertion
	for (int i = 1; i <= n1; i++)
		for (int j = 1; j <= n2; j++)
			dp[i][j] = min(
				1 + min(dp[i - 1][j], dp[i][j - 1]),
				dp[i - 1][j - 1] + (w1[i - 1] != w2[j - 1])
			);
	return dp[n1][n2];
}
\end{lstlisting}

\subsection{Knuth-Morris-Pratt algorithm $\mathcal{O}(N + M)$}

\begin{lstlisting}
int kmp_search(const string &word, const string &text) {
	int n = word.size();
	vi T(n + 1, 0);
	for (int i = 1, j = 0; i < n; ) {
		if (word[i] == word[j]) T[++i] = ++j; // match
		else if (j > 0) j = T[j]; // fallback
		else i++; // no match, keep zero
	}
	int matches = 0;
	for (int i = 0, j = 0; i < text.size(); ) {
		if (text[i] == word[j]) {
			i++;
			if (++j == n) { // match at interval [i - n, i)
				matches++;
				j = T[j];
			}
		} else if (j > 0) j = T[j];
		else i++;
	}
	return matches;
}
\end{lstlisting}

\subsection{Aho-Corasick Algorithm $\mathcal{O}(N + \sum_{i=1}^{m} |S_i|)$}

All given P must be unique!

\begin{lstlisting}
const int MAXP = 100, MAXLEN = 200, SIGMA = 26, MAXTRIE = MAXP * MAXLEN;

int nP;
string P[MAXP], S;

int pnr[MAXTRIE], to[MAXTRIE][SIGMA], sLink[MAXTRIE], dLink[MAXTRIE], nnodes;

void ahoCorasick() {
	fill_n(pnr, MAXTRIE, -1);
	for (int i = 0; i < MAXTRIE; i++) fill_n(to[i], SIGMA, 0);
	fill_n(sLink, MAXTRIE, 0);
	fill_n(dLink, MAXTRIE, 0);
	nnodes = 1;
	// STEP 1: MAKE A TREE
	for (int i = 0; i < nP; i++) {
		int cur = 0;
		for (char c : P[i]) {
			int i = c - 'a';
			if (to[cur][i] == 0) to[cur][i] = nnodes++;
			cur = to[cur][i];
		}
		pnr[cur] = i;
	}
	// STEP 2: CREATE SUFFIX_LINKS AND DICT_LINKS
	queue<int> q;
	q.push(0);
	while (!q.empty()) {
		int cur = q.front();
		q.pop();
		for (int c = 0; c < SIGMA; c++) {
			if (to[cur][c]) {
				int sl = sLink[to[cur][c]] = cur == 0 ? 0 : to[sLink[cur]][c];
				// if all strings have equal length, remove this:
				dLink[to[cur][c]] = pnr[sl] >= 0 ? sl : dLink[sl];
				q.push(to[cur][c]);
			} else to[cur][c] = to[sLink[cur]][c];
		}
	}
	// STEP 3: TRAVERSE S
	for (int cur = 0, i = 0, n = S.size(); i < n; i++) {
		cur = to[cur][S[i] - 'a'];
		for (int hit = pnr[cur] >= 0 ? cur : dLink[cur]; hit; hit = dLink[hit]) {
			cerr << P[pnr[hit]] << " found at [" << (i + 1 - P[pnr[hit]].size()) << ", " << i << "]" << endl;
		}
	}
}
\end{lstlisting}

\section{Geometry}

\begin{lstlisting}
const double EPS = 1e-7, PI = acos(-1.0);

typedef long long NUM; // EITHER double OR long long
typedef pair<NUM, NUM> pt;
#define x first
#define y second

pt operator+(pt p, pt q) { return pt(p.x + q.x, p.y + q.y); }
pt operator-(pt p, pt q) { return pt(p.x - q.x, p.y - q.y); }

pt& operator+=(pt &p, pt q) { return p = p + q; }
pt& operator-=(pt &p, pt q) { return p = p - q; }

pt operator*(pt p, NUM l) { return pt(p.x * l, p.y * l); }
pt operator/(pt p, NUM l) { return pt(p.x / l, p.y / l); }

NUM operator*(pt p, pt q) { return p.x * q.x + p.y * q.y; }
NUM operator^(pt p, pt q) { return p.x * q.y - p.y * q.x; }

istream& operator>>(istream &in, pt &p) { return in >> p.x >> p.y; }
ostream& operator<<(ostream &out, pt p) { return out << '(' << p.x << ", " << p.y << ')'; }

NUM lenSq(pt p) { return p * p; }
NUM lenSq(pt p, pt q) { return lenSq(p - q); }
double len(pt p) { return hypot(p.x, p.y); } // more overflow safe
double len(pt p, pt q) { return len(p - q); }

typedef pt frac;
typedef pair<double, double> vec;
vec getvec(pt p, pt dp, frac t) { return vec(p.x + 1. * dp.x * t.x / t.y, p.y + 1. * dp.y * t.x / t.y); }

// square distance from pt a to line bc
frac distPtLineSq(pt a, pt b, pt c) {
	a -= b, c -= b;
	return frac((a ^ c) * (a ^ c), c * c);
}

// square distance from pt a to linesegment bc
frac distPtSegmentSq(pt a, pt b, pt c) {
	a -= b; c -= b;
	NUM dot = a * c, len = c * c;
	if (dot <= 0) return frac(a * a, 1);
	if (dot >= len) return frac((a - c) * (a - c), 1);
	return frac(a * a * len - dot * dot, len);
}

// projects pt a onto linesegment bc
frac proj(pt a, pt b, pt c) { return frac((a - b) * (c - b), (c - b) * (c - b)); }
vec projv(pt a, pt b, pt c) { return getvec(b, c - b, proj(a, b, c)); }

bool collinear(pt a, pt b, pt c) { return ((a - b) ^ (a - c)) == 0; }

bool pointOnSegment(pt a, pt b, pt c) {
	NUM dot = (a - b) * (c - b), len = (c - b) * (c - b);
	return collinear(a, b, c) && 0 <= dot && dot <= len;
}

// true => 1 intersection, false => parallel, so 0 or \infty solutions
bool linesIntersect(pt a, pt b, pt c, pt d) { return ((a - b) ^ (c - d)) != 0; }
vec lineLineIntersection(pt a, pt b, pt c, pt d) {
	double det = (a - b) ^ (c - d);
	pt ret = (c - d) * (a ^ b) - (a - b) * (c ^ d);
	return vec(ret.x / det, ret.y / det);
}

// dp, dq are directions from p, q
// intersection at p + t_i dp, for 0 <= i < return value
int segmentIntersection(pt p, pt dp, pt q, pt dq, frac &t0, frac &t1)
{
	if (dp * dp == 0) swap(p, q), swap(dp, dq); // dq = 0
	if (dp * dp == 0) { t0 = t1 = frac(0, 1); return p == q; } // dp = dq = 0

	pt dpq = (q - p);
	NUM c = dp ^ dq, c0 = dpq ^ dp, c1 = dpq ^ dq;
	if (c == 0) { // parallel, dp > 0, dq >= 0
		if (c0 != 0) return 0; // not collinear
		NUM v0 = dpq * dp, v1 = v0 + dq * dp, dp2 = dp * dp;
		if (v1 < v0) swap(v0, v1);
		t0 = frac(v0 = max(v0, (NUM) 0), dp2);
		t1 = frac(v1 = min(v1, dp2), dp2);
		return (v0 <= v1) + (v0 < v1);
	} else if (c < 0) c = -c, c0 = -c0, c1 = -c1;
	t0 = t1 = frac(c1, c);
	return 0 <= min(c0, c1) && max(c0, c1) <= c;
}

// Returns TWICE the area of a polygon to keep it an integer
NUM polygonTwiceArea(const vector<pt> &pts) {
	NUM area = 0;
	for (int N = pts.size(), i = 0, j = N - 1; i < N; j = i++)
		area += pts[i] ^ pts[j];
	return abs(area); // area < 0 <=> pts ccw
}

bool pointInPolygon(pt p, const vector<pt> &pts) {
	double sum = 0;
	for (int N = pts.size(), i = 0, j = N - 1; i < N; j = i++) {
		if (pointOnSegment(p, pts[i], pts[j])) return true; // boundary
		double angle = acos((pts[i] - p) * (pts[j] - p) / len(pts[i], p) / len(pts[j], p));
		sum += ((pts[i] - p) ^ (pts[j] - p)) < 0 ? angle : -angle;
	}
	return abs(abs(sum) - 2 * PI) < EPS;
}
\end{lstlisting}

\begin{comment}
// Assumption: polygon has unique points
// 0: no, 1: yes, 2: on boundary
int pointInConvex(pt p, const vector<pt> &pts) {
bool onBoundary = false;
for (int N = pts.size(), sgn = 0, i = 0, j = N - 1; i < N; j = i++) {
NUM cross = (pts[i] - p) ^ (pts[j] - p);
int s = (cross > 0) - (cross < 0);
if (cross == 0) onBoundary = true;
else if (sgn == 0) sgn = s;
else if (sgn != s) return 0;
}
return onBoundary ? 2 : 1;
}
\end{comment}

\subsection{Convex Hull $\mathcal{O}(n \log n)$}

\begin{lstlisting}
// points are given by: pts[ret[0]], pts[ret[1]], ... pts[ret[ret.size()-1]]
vi convexHull(const vector<pt> &pts) {
	if (pts.empty()) return vi();
	vi ret;
	// find one outer point:
	int fsti = 0, n = pts.size();
	pt fstpt = pts[0];
	for(int i = n; i--; ) {
		if (pts[i] < fstpt) fstpt = pts[fsti = i];
	}
	ret.pb(fsti);
	pt refr = pts[fsti];

	vi ord; // index into pts
	for (int i = n; i--; ) {
		if (pts[i] != refr) ord.pb(i);
	}
	sort(ord.begin(), ord.end(), [&pts, &refr] (int a, int b) -> bool {
		NUM cross = (pts[a] - refr) ^ (pts[b] - refr);
		return cross != 0 ? cross > 0 : lenSq(refr, pts[a]) < lenSq(refr, pts[b]);
	});
	for (int i : ord) {
		// NOTE: > INCLUDES points on the hull-line, >= EXCLUDES
		while (ret.size() > 1 &&
				((pts[ret[ret.size()-2]]-pts[ret.back()]) ^ (pts[i]-pts[ret.back()])) >= 0)
			ret.pop_back();
		ret.pb(i);
	}
	return ret;
}
\end{lstlisting}

\subsection{Rotating Calipers $\mathcal{O}(n)$}

Finds the longest distance between two points in a convex hull.

\begin{lstlisting}
NUM rotatingCalipers(vector<pt> &hull) {
	int n = hull.size(), a = 0, b = 1;
	if (n <= 1) return 0.0;
	while (((hull[1] - hull[0]) ^ (hull[(b + 1) % n] - hull[b])) > 0) b++;
	NUM ret = 0.0;
	while (a < n) {
		ret = max(ret, lenSq(hull[a], hull[b]));
		if (((hull[(a + 1) % n] - hull[a % n]) ^ (hull[(b + 1) % n] - hull[b])) <= 0) a++;
		else if (++b == n) b = 0;
	}
	return ret;
}
\end{lstlisting}

\subsection{Closest points $\mathcal{O}(n \log n)$}

\begin{lstlisting}
int n;
pt pts[maxn];

struct byY {
	bool operator()(int a, int b) const { return pts[a].y < pts[b].y; }
};

inline NUM dist(pii p) {
	return hypot(pts[p.x].x - pts[p.y].x, pts[p.x].y - pts[p.y].y);
}

pii minpt(pii p1, pii p2) {
	return (dist(p1) < dist(p2)) ? p1 : p2;
}

// closest pts (by index) inside pts[l ... r], with sorted y values in ys
pii closest(int l, int r, vi &ys) {
	if (r - l == 2) { // don't assume 1 here.
		ys = { l, l + 1 };
		return pii(l, l + 1);
	} else if (r - l == 3) { // brute-force
		ys = { l, l + 1, l + 2 };
		sort(ys.begin(), ys.end(), byY());
		return minpt(pii(l, l + 1), minpt(pii(l, l + 2), pii(l + 1, l + 2)));
	}
	int m = (l + r) / 2;
	vi yl, yr;
	pii delta = minpt(closest(l, m, yl), closest(m, r, yr));
	NUM ddelta = dist(delta), xm = .5 * (pts[m-1].x + pts[m].x);
	merge(yl.begin(), yl.end(), yr.begin(), yr.end(), back_inserter(ys), byY());
	deque<int> q;
	for (int i : ys) {
		if (abs(pts[i].x - xm) <= ddelta) {
			for (int j : q) delta = minpt(delta, pii(i, j));
			q.pb(i);
			if (q.size() > 8) q.pop_front(); // magic from Introduction to Algorithms.
		}
	}
	return delta;
}
\end{lstlisting}

\section{Miscellaneous}
\subsection{Binary search $\mathcal{O}(\log (hi - lo))$}

% \noindent\begin{minipage}{.45\textwidth}
% [title={Inclusive, Exclusive}]
\begin{lstlisting}
bool test(int n);

int search(int lo, int hi) {
	// assert(test(lo) && !test(hi));
	while (hi - lo > 1) {
		int m = (lo + hi) / 2;
		(test(m) ? lo : hi) = m;
	}
	// assert(test(lo) && !test(hi));
	return lo;
}
\end{lstlisting}

\begin{comment}
\end{minipage}\hfill
\begin{minipage}{.45\textwidth}
	\begin{lstlisting}[title={Inclusive, Inclusive}]
bool test(int n);
	
int lo = 0, hi = n - 1;
// assert(test(lo) && !test(hi + 1));
while (lo < hi) {
	int mid = (lo + hi + 1) / 2;
	if (test(mid)) lo = mid;
	else hi = mid - 1;
}
return lo;
	\end{lstlisting}
\end{minipage}
\end{comment}

\subsection{Fast Fourier Transform $\mathcal{O}(n \log n)$}

Given two polynomials $A(x) = a_0 + a_1 x + \dots + a_{n/2} x^{n/2}$ and $B(x) = b_0 + b_1 x + \dots + b_{n/2} x^{n/2}$, FFT calculates all coefficients of $C(x) = A(x) \cdot B(x) = c_0 + c_1 x + \dots c_{n} x^{n}$, with $c_i = \sum_{j=0}^{i} a_j b_{i-j}$.

\begin{lstlisting}
typedef complex<double> cpx;
const int logmaxn = 20, maxn = 1 << logmaxn;

cpx a[maxn] = {}, b[maxn] = {}, c[maxn];

void fft(cpx *src, cpx *dest) {
	for (int i = 0, rep = 0; i < maxn; i++, rep = 0) {
		for (int j = i, k = logmaxn; k--; j >>= 1) rep = (rep << 1) | (j & 1);
		dest[rep] = src[i];
	}
	for (int s = 1, m = 1; m <= maxn; s++, m *= 2) {
		cpx r = exp(cpx(0, 2.0 * PI / m));
		for (int k = 0; k < maxn; k += m) {
			cpx cr(1.0, 0.0);
			for (int j = 0; j < m / 2; j++) {
				cpx t = cr * dest[k + j + m / 2];
				dest[k + j + m / 2] = dest[k + j] - t;
				dest[k + j] += t;
				cr *= r;
			}
		}
	}
}

void multiply() {
	fft(a, c);
	fft(b, a);
	for (int i = 0; i < maxn; i++) b[i] = conj(a[i] * c[i]);
	fft(b, c);
	for (int i = 0; i < maxn; i++) c[i] = conj(c[i]) / (1.0 * maxn);
}
\end{lstlisting}

\subsection{Minimum Assignment (Hungarian Algorithm) $\mathcal{O}(n^{3})$}

\begin{lstlisting}
int a[MAXN + 1][MAXM + 1]; // matrix, 1-based

int minimum_assignment(int n, int m) { // n rows, m columns
	vi u(n + 1), v(m + 1), p(m + 1), way(m + 1);

	for (int i = 1; i <= n; i++) {
		p[0] = i;
		int j0 = 0;
		vi minv(m + 1, INF);
		vector<char> used(m + 1, false);
		do {
			used[j0] = true;
			int i0 = p[j0], delta = INF, j1;
			for (int j = 1; j <= m; j++)
				if (!used[j]) {
					int cur = a[i0][j] - u[i0] - v[j];
					if (cur < minv[j]) minv[j] = cur, way[j] = j0;
					if (minv[j] < delta) delta = minv[j], j1 = j;
				}
			for (int j = 0; j <= m; j++) {
				if(used[j]) u[p[j]] += delta, v[j] -= delta;
				else minv[j] -= delta;
			}
			j0 = j1;
		} while (p[j0] != 0);
		do {
			int j1 = way[j0];
			p[j0] = p[j1];
			j0 = j1;
		} while (j0);
	}

	// column j is assigned to row p[j]
	// for (int j = 1; j <= m; ++ j) ans[p[j]] = j;
	return -v[0];
}
\end{lstlisting}

\subsection{Partial linear equation solver $\mathcal{O}(N^3)$}

\begin{lstlisting}
typedef double NUM;

#define MAXN 110
#define EPS 1e-5

NUM mat[MAXN][MAXN + 1], vals[MAXN];
bool hasval[MAXN];

bool is_zero(NUM a) { return -EPS < a && a < EPS; }
bool eq(NUM a, NUM b) { return is_zero(a - b); }

int solvemat(int n)
{
	for(int i = 0; i < n; i++)
		for (int j = 0; j < n; j++) cin >> mat[i][j];
	for (int i = 0; i < n; i++) cin >> mat[i][n];

	int pivrow = 0, pivcol = 0;
	while (pivcol < n) {
		int r = pivrow, c;
		while (r < n && is_zero(mat[r][pivcol])) r++;
		if (r == n) { pivcol++; continue; }

		for (c = 0; c <= n; c++) swap(mat[pivrow][c], mat[r][c]);

		r = pivrow++; c = pivcol++;
		NUM div = mat[r][c];
		for (int col = c; col <= n; col++) mat[r][col] /= div;
		for (int row = 0; row < n; row++) {
			if (row == r) continue;
			NUM times = -mat[row][c];
			for (int col = c; col <= n; col++) mat[row][col] += times * mat[r][col];
		}
	}
	// now mat is in RREF
	for (int r = pivrow; r < n; r++)
		if (!is_zero(mat[r][n])) return 0;

	fill_n(hasval, n, false);
	for (int col = 0, row; col < n; col++) {
		hasval[col] = !is_zero(mat[row][col]);
		if (!hasval[col]) continue;
		for (int c = col + 1; c < n; c++) {
			if (!is_zero(mat[row][c])) hasval[col] = false;
		}
		if (hasval[col]) vals[col] = mat[row][n];
		row++;
	}

	for (int i = 0; i < n; i++)
		if (!hasval[i]) return 2;
	return 1;
}
\end{lstlisting}
\end{multicols*}
\end{document}
